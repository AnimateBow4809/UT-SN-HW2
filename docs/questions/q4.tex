
\begin{question}
	\questiontext{Structural Analysis of Political Power}
	In this assignment, you will analyze the interaction network (mutual likes) of 5,768 politicians worldwide on Facebook. This network is an Induced Subgraph containing exclusively political nodes. You must transition univariate statistical analysis to deep structural analysis to demonstrate how a politician’s topological position determines their real-world role.
	\begin{subquestion}{Power Geometry (Quantity, Quality, and Access)}
		Power in networks manifests in three forms: the volume of connections (Quantity), the importance of connections (Quality), and the speed of access to the entire network (Access).
		\begin{enumerate}
			\item  Centrality Calculations: Calculate Normalized Degree , Eigenvector Centrality , and Closeness Centrality for all. Extract and report the top 10 nodes for each metric.
			\answer{
				 We computed the \emph{Normalized Degree Centrality}, \emph{Eigenvector Centrality}, and \emph{Closeness Centrality} for all nodes in the network. The top 10 nodes for each metric are reported below.
				
				\begin{itemize}
					\item \textbf{Top 10 by Normalized Degree Centrality:}
					\begin{enumerate}
						\item Manfred Weber
						\item Joachim Herrmann
						\item Katarina Barley
						\item Arno Klare MdB
						\item Katja Mast
						\item Barack Obama
						\item Angela Merkel
						\item Niels Annen
						\item Martin Schulz
						\item Sir Peter Bottomley MP
					\end{enumerate}
					
					\item \textbf{Top 10 by Eigenvector Centrality:}
					\begin{enumerate}
						\item Katarina Barley
						\item Arno Klare MdB
						\item Katja Mast
						\item Christian Petry
						\item Heike Baehrens
						\item Klaus Mindrup
						\item Michelle Müntefering
						\item Niels Annen
						\item Johannes Schraps
						\item Sigmar Gabriel
					\end{enumerate}
					
					\item \textbf{Top 10 by Closeness Centrality:}
					\begin{enumerate}
						\item Barack Obama
						\item Michael Roth
						\item Niels Annen
						\item Tanja Fajon
						\item Malcolm Turnbull
						\item Mariano Rajoy Brey
						\item Achim Post
						\item Peter Tauber
						\item Dietmar Nietan
						\item Hillary Clinton
					\end{enumerate}
				\end{itemize}				
			}
			\item  Gap Analysis: Generate a Scatter Plot with Degree on the X-axis and Eigenvector on the Y-axis. Identify nodes that deviate significantly positively from the correlation line (Low Degree but High Eigenvector).
			\answer{
			\textbf{Gap Analysis:} We performed a gap analysis by generating a scatter plot shown in Figure~\ref{fig:q4_a_plot} with \emph{Normalized Degree Centrality} on the X-axis and \emph{Eigenvector Centrality} on the Y-axis. A linear regression line was fitted to capture the overall correlation between quantity of connections (degree) and quality of connections (eigenvector).
			
			Nodes that lie significantly \emph{above} the regression line exhibit \textbf{high eigenvector centrality despite relatively low degree}, indicating disproportionate structural influence or \emph{power efficiency}. These nodes are identified via large positive residuals from the regression.
			
			The top 10 positive deviants (low degree / high eigenvector) are:
			\begin{enumerate}
				\item Christian Petry
				\item Heike Baehrens
				\item Katarina Barley
				\item Klaus Mindrup
				\item Arno Klare MdB
				\item Michelle Müntefering
				\item Katja Mast
				\item Sigmar Gabriel
				\item Johannes Schraps
				\item Wolfgang Hellmich
			\end{enumerate}
			
			These actors occupy strategically influential positions by being strongly connected to highly central nodes rather than maintaining a large number of connections. This highlights a distinction between \emph{connection volume} and \emph{connection quality}, which is not captured by degree centrality alone.
			}
			\begin{figure}[H]
				\centering
				\includegraphics[width=0.85\textwidth,keepaspectratio]
				{imgs/q4_a_plot.png}
				\caption{scatter plot \emph{Normalized Degree Centrality} vs \emph{Eigenvector Centrality} showing positive deviants.}
				\label{fig:q4_a_plot}
			\end{figure}
			\item  Three-Way Case Study: Select three politicians who exhibit Low Degree (ranked outside the top 100) but High Eigenvector (ranked within the top 50). Closeness Analysis: Examine the Closeness rank of these three individuals Role Analysis: Investigate the real-world names and positions of these selected
			individuals. Does the mathematical analysis corroborate their actual roles (e.g.,
			Chief of Staff, Executive Secretary, or Senior Advisor)?
			\answer{
			 \textbf{Three-Way Case Study:} To connect structural metrics with real-world roles, we selected three politicians who exhibit \emph{Low Degree Centrality} (ranked outside the top 100) but \emph{High Eigenvector Centrality} (ranked within the top 50). This combination highlights actors who are not broadly connected, yet are embedded within highly influential neighborhoods of the network.
			
			\textbf{Interpretive Framework (Closeness Centrality):}
			\begin{itemize}
				\item \textbf{High Closeness:} The individual resides in the \emph{geometric heart} of the network, acting with relative independence and direct access to many parts of the graph.
				\item \textbf{Low Closeness:} The individual is \emph{structurally dependent} on powerful neighbors, indicating a marginal attachment to the core and influence mediated through elite ties.
			\end{itemize}
			
			\begin{enumerate}
				\item \textbf{Carsten Schneider} (\( \text{Degree Rank} = 228.5,\ \text{Eigenvector Rank} = 43,\ \text{Closeness Rank} = 29 \))  
				Schneider’s high closeness places him near the geometric center of the network, suggesting operational independence despite a limited number of direct connections. This aligns with his real-world role as a senior parliamentary coordinator in the German Bundestag, where influence stems from central positioning rather than high interaction volume.
				
				\item \textbf{Bernd Lange} (\( \text{Degree Rank} = 118,\ \text{Eigenvector Rank} = 44,\ \text{Closeness Rank} = 14 \))  
				Lange exhibits both high eigenvector and very high closeness, indicating strong central embeddedness and autonomy. This corroborates his role as Chair of the European Parliament’s Committee on International Trade, a position that naturally situates him at the network’s core with direct access to multiple influential actors.
				
				\item \textbf{Barbara Hendricks} (\( \text{Degree Rank} = 130.5,\ \text{Eigenvector Rank} = 33,\ \text{Closeness Rank} = 283.5 \))  
				In contrast, Hendricks shows low closeness despite high eigenvector centrality. This pattern indicates reliance on a small set of powerful neighbors rather than broad network accessibility. Such a structure is consistent with her status as a former Federal Minister, where residual influence is maintained through elite, high-impact ties rather than central operational presence.
			\end{enumerate}
			
			\textbf{Conclusion:}  
			Incorporating closeness centrality sharpens the interpretation of influence. High eigenvector centrality alone captures \emph{who one is connected to}, while closeness reveals \emph{how independently} that influence is exercised. The three cases demonstrate two distinct modes of power: centrally embedded autonomous actors versus peripheral elites whose influence is mediated through powerful neighbors—both of which are accurately reflected by the network metrics.
			}
		\end{enumerate}
	\end{subquestion}
\end{question}