
\begin{question}
	\questiontext{Distances and Neighbors}
	\begin{subquestion}{Consider the below network.}
		For each of the following scenarios, indicate which node would be the best choice, giving
		reasons:
		\begin{enumerate}
			\item The mayor wants to install a radio broadcast station so that, in a crisis, a single
			nationwide message can reach all areas. The goal is that every node’s distance to
			the station (independently of other nodes) is as small as possible —in other words,
			the maximum distance from any node to the station should be minimized.
			\answer{
				To determine the optimal location for the radio station, we identify the \textbf{Graph Center} by computing the \textbf{Eccentricity} of each node. The eccentricity of a node is the distance to the farthest node in the network; minimizing this ensures that the maximum travel time for a broadcast signal is as small as possible.  
				
				The eccentricity vector $\epsilon$ for nodes $1$ through $15$ is:
				\[
				\epsilon = \begin{bmatrix} 
					4.0 \\ 3.0 \\ 4.0 \\ 5.0 \\ 4.0 \\ 4.0 \\ 4.0 \\ 5.0 \\ 4.0 \\ 4.0 \\ 4.0 \\ 5.0 \\ 3.0 \\ 4.0 \\ 5.0
				\end{bmatrix}
				\]
				
				
				The minimum eccentricity is $\min(\epsilon) = \mathbf{3.0}$, which occurs at \textbf{Node 2} and \textbf{Node 13}. Therefore, placing the radio station at either of these nodes ensures that every other node is reachable within at most 3 hops, providing the most efficient coverage across the network.
			}
			
			\item  The mayor wants to choose a node for a bookstore so that the sum of distances from 
			all residents to that node is minimized.
			\answer{
				To determine the optimal node for placing the bookstore, we use \textbf{closeness centrality}, 
				which measures how close a node is on average to all other nodes in the graph. 
				By choosing the node with the highest closeness centrality, we ensure that the nodes can reach 
				the bookstore as quickly as possible on average.
				
				The formula for closeness centrality of a node $v$ is:
				\[
				C(v) = \frac{n-1}{\sum_{u \in V} d(v,u)}
				\]
				where $d(v,u)$ is the shortest-path distance between nodes $v$ and $u$, and $n$ is the total number of nodes.
				
				For our 15-node graph, the closeness centrality values are:
				
				\[
				\begin{bmatrix}
					0.4667 & 0.4667 & 0.4667 & 0.4 & 0.4516 \\
					0.4516 & 0.4667 & 0.4242 & 0.4516 & 0.4375 \\
					0.4118 & 0.4 & 0.5185 & 0.3684 & 0.2745
				\end{bmatrix}
				\]
				
				Node $i=13$ (value 0.5185) has the highest closeness centrality, so we choose it as the location 
				for the bookstore to ensure efficient distance to other nodes.
			}
			\item Two stores have decided to open new branches in the city. Each person (node)
			buys from the nearest store. If a person is at equal distance from both stores, their
			purchases are split equally between them. First, select the best node to open store
			A, then determine the best location for store B given that choice.
			\answer{
				From the closeness centrality analysis, we know that node 13 has the highest value, 
				so it is the best choice for opening store A. Given store A at node 13, the nodes 
				that would maximize the number of customers for store B are those with the highest 
				customer counts. In this case, nodes 1, 3, 7, and 8 each attract 7.0 customers, 
				making them all equally good candidates for store B.
				
				The number of customers each node would get if store A is at node 13 is shown below, 
				with the best candidates for store B highlighted in bold:
				\[
				\begin{bmatrix}
					1 & \mathbf{7.0} \\
					2 & 6.0 \\
					3 & \mathbf{7.0} \\
					4 & 6.5 \\
					5 & 6.5 \\
					6 & 5.5 \\
					7 & \mathbf{7.0} \\
					8 & \mathbf{7.0} \\
					9 & 6.5 \\
					10 & 5.0 \\
					11 & 5.5 \\
					12 & 6.5 \\
					14 & 2.0 \\
					15 & 1.5
				\end{bmatrix}
				\]
				Therefore, the recommended locations are: store A at node 13, and store B at any 
				of nodes 1, 3, 7, or 8.
			}

		\end{enumerate}
	\end{subquestion}
	\begin{subquestion}{Consider the below network.}
		\begin{enumerate}
			\item What is the closeness centrality for node C ?
			\answer{
			The closeness centrality(both standard and normalized) for node c are as follows:
			\[\begin{gathered}
				{C_C}(c) = \frac{1}{{\sum\limits_{j = 1}^n {{d_{cj}}} }} = \frac{1}{{n - 1}} \hfill \\
				{{\tilde C}_C}(c) = \frac{{n - 1}}{{\sum\limits_{j = 1}^n {{d_{cj}}} }} = \frac{{n - 1}}{{n - 1}} = 1 \hfill \\ 
			\end{gathered} \]
			}
			\item  Derive the closeness centrality value for the nodes on the ring as a function of n
			\answer{
			The closeness centrality(both standard and normalized) are as follows:
			\[\begin{gathered}
				{C_C}({n_i}) = \frac{1}{{\sum\limits_{j = 1}^n {{d_{cj}}} }} = \frac{1}{{1 + 1 + 1 + (n - 4)(2)}} = \frac{1}{{2n - 5}} \hfill \\
				{{\tilde C}_C}({n_i}) = \frac{{n - 1}}{{\sum\limits_{j = 1}^n {{d_{cj}}} }} = \frac{{n - 1}}{{1 + 1 + 1 + (n - 4)(2)}} = \frac{{n - 1}}{{2n - 5}} \hfill \\ 
			\end{gathered} \]
			}
		\end{enumerate}
	\end{subquestion}
\end{question}